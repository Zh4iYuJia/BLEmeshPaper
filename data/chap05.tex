\chapter{基于BLE Mesh的智能家居平台实现}
\section{开发语言的选择}
现今,市面上有许多开发语言可供选择,它们各自有各自的优点,如C/C++语言执行速度快,程序占用空间小;Go语言代码简洁,执行速度快;Dart语言易于编写,使用Flutter框架后还可实现一套代码就能编译出iOS/Android端的App。

对于本项目的BLE Mesh节点而言,其性能不高且存储空间狭小,并且需使用Nordic芯片厂商所提供的nRF5 SDK for Mesh,因此选择了C语言作为开发语言。

对于本项目的网关端而言,其性能充足但需要高扩展性且逻辑复杂,为了开发方便,因此选择了Go语言作为开发语言。

对于本项目的移动端而言,需要同时支持Android与iOS系统,为了界面逻辑统一,因此选择了Dart语言与Flutter框架作为开发语言。

\section{集成开发环境的选择}
在BLE Mesh节点开发上,目前针对嵌入式开发并支持C语言主要是SEGGER Embedded Studio及Keil,SEGGER Embedded Studio相较于Keil更加易用,并且能很好地兼容nRF5 SDK for Mesh,因此选择了SEGGER Embedded Studio作为BLE Mesh节点的集成开发环境。

在网关端开发上,目前针对Go语言的集成开发环境主要有GoLand与VS Code,但GoLand功能更为强大,能很好地进行调试以及包管理,因此选择了GoLand作为网关端的集成开发环境。

在移动端开发上,目前Google推荐使用Android Studio进行Dart+Flutter的开发,因此选择了Android Studio作为移动端的集成开发环境。
\section{逻辑部分的实现}
\subsection{BLE Mesh节点}
内容
\subsubsection{Server}
\subsubsection{Provisioner}

\subsection{网关端}
内容

\subsection{移动端}
内容
