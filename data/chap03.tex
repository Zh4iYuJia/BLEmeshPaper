\chapter{采用的技术手段}

\section{蓝牙低功耗}
蓝牙低功耗(Bluetooth Low Energy,或称Bluetooth LE、BLE,旧商标Bluetooth Smart)也称蓝牙低能耗、低功耗蓝牙,是蓝牙技术联盟设计和销售的一种个人局域网技术,旨在用于医疗保健、运动健身、信标、安防、家庭娱乐等领域的新兴应用。相较经典蓝牙,低功耗蓝牙旨在保持同等通信范围的同时显著降低功耗和成本。\cite{ble}此外,蓝牙低功耗技术还可在传输过程中采用AES-128-CBC加密,在一定程度上保证了传输过程中的安全性。

本项目使用蓝牙低功耗技术实现多个智能家居节点之间及单个智能家居节点与手机的连接。其低功耗的特点,使得节点只需要一枚纽扣电池即可工作很长时间;传输过程中自动加密的特性也使得整个智能家居平台中的安全性得到了一定的保障。

\section{网状网络}
网状网络(英文:Mesh Network)是一种在网络节点间透过动态路由的方式来进行资料与控制指令的传送。这种网络可以保持每个节点间的连线完整,当网络拓扑中有某节点失效或无法服务时,这种架构允许使用“跳跃”的方式形成新的路由后将讯息送达传输目的地。在网状网络中,所有节点都可与拓扑中所有节点进行连线而形成一个“局域网络”。网状网络与一般网络架构的差异处在于,所有节点可以透过多次跳跃进行数据通信,但它们通常不是移动式装置。网状网络可以视为是一种点对点的架构。移动式点对点网络与网状网络在架构上是非常相似的,只是移动式点对点网络还必须随时更新组态以因应各节点移动的情形。\cite{mesh}

本项目使用网状网络技术增强了智能家居节点布局的灵活性,在部署新节点时只需保证其与现有的任意节点能建立连接即可。移动端也可接入到任意节点,实现从单点控制整个网状网络。