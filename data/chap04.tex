\chapter{基于BLE Mesh的智能家居平台功能设计}

\section{架构设计}
本项目主要分为三个部分:BLE Mesh节点,网关端及移动端,如图~\ref{fig:arch}。

BLE Mesh节点主要实现传感器、继电器、灯泡等设备的接入,并且节点间可自动建立BLE协议下的Mesh网络,使得用户即使不使用网关也能用移动端软件连接至Mesh网络控制所有节点。

网关端主要实现设备层与管理应用层的连接。采用插件化的方案,使得有能力的爱好者或其他智能设备厂商,可以自己根据本项目简单的API来制作插件接入本平台;也便于接入到其他的管理平台,例如Apple HomeKit、Google Home等,让用户不需要下载其他程序就能管理网关下的所有设备。

移动端主要实现控制Mesh网络或网关下的设备、添加自动化的功能。
\begin{figure}[H]
    \centering
    \includegraphics{arch_basic.pdf}
    \caption{基于BLE Mesh的智能家居平台基本架构图}
    \label{fig:arch}
\end{figure}

\section{BLE Mesh节点}
\label{design:blemeshnode}
BLE Mesh节点不同于一般智能家居设备,它们之间通过另一个BLE Mesh分发节点或手机分发配置后构成网状网络,之后所有数据就可通过其他节点转发至目标设备。

每个节点均可接入多个模块,例如继电器、开关、传感器等。不需要每个模块都单独配一套蓝牙通讯装置,节约了成本也节约了频段资源。

网关或移动端设备也可以通过任一节点代理入Mesh网络,之后便可以访问所有的节点及其连接的模块。
\begin{figure}[H]
    \centering
    \includegraphics{arch_mesh.pdf}
    \caption{基于BLE Mesh的智能家居平台BLE Mesh网络架构图}
    \label{fig:arch}
\end{figure}

\section{网关端}
网关端实现了设备层与管理应用层的连接,即将不同的智能设备(如~\ref{design:blemeshnode}~章所提到的BLE Mesh节点等)与不同的管理平台(如~\ref{design:mobile}~章所提到的基于BLE Mesh的智能家居平台移动端、Apple HomeKit等)相连接。

网关端还创新性地采用了插件化的方案,使得爱好者或智能家居厂商可自行按照网关端插件API编写对应插件,让家中所有的智能设备最终能在一个平台上进行管理。目前本项目已开发的插件有BLE Mesh插件(用于连接BLE Mesh节点)、本项目移动端所使用的WebSocket插件(将设备以WebSocket的方式暴露到内网便于移动端连接)以及Apple HomeKit插件(将设备以符合Apple HomeKit方式暴露到内网便于苹果设备上的“家庭”应用连接)。
\begin{figure}[H]
    \centering
    \includegraphics{arch_gateway.pdf}
    \caption{基于BLE Mesh的智能家居平台网关端架构图}
    \label{fig:arch}
\end{figure}

\section{移动端}
\label{design:mobile}
移动端包含了UI层、设备层及自动化层。UI层支持用户查看所有设备的状态、控制所有设备及添加自动化流程。设备层支持通过单个BLE Mesh节点代理入网的方式及通过连接至网关端WebSocket插件的方式访问整个智能家居网络。而自动化层支持用户在本机执行自动化流程或将自动化流程提交到网关。
\begin{figure}[H]
    \centering
    \includegraphics{arch_mobile.pdf}
    \caption{基于BLE Mesh的智能家居平台移动端架构图}
    \label{fig:arch}
\end{figure}
