\thusetup{
  %******************************
  % 注意:
  %   1. 配置里面不要出现空行
  %   2. 不需要的配置信息可以删除
  %******************************
  %
  %=====
  % 秘级
  %=====
  % secretlevel={秘密},
  % secretyear={10},
  %
  %=========
  % 中文信息
  %=========
  ctitle={基于BLE Mesh的智能家居平台\\研究报告},
  cdepartment={金苹果锦城第一中学},
  cmajor={计算机科学与技术},
  cauthor={李董睿煊、翟羽佳},
  csupervisor={伍冬莉},
  % 日期自动使用当前时间,若需指定按如下方式修改:
  % cdate={超新星纪元},
  %
  % 博士后专有部分
  % catalognumber     = {分类号},  % 可以留空
  % udc               = {UDC},  % 可以留空
  % id                = {编号},  % 可以留空: id={},
  % cfirstdiscipline  = {计算机科学与技术},  % 流动站(一级学科)名称
  % cseconddiscipline = {系统结构},  % 专 业(二级学科)名称
  % postdoctordate    = {2009 年 7 月——2011 年 7 月},  % 工作完成日期
  % postdocstartdate  = {2009 年 7 月 1 日},  % 研究工作起始时间
  % postdocenddate    = {2011 年 7 月 1 日},  % 研究工作期满时间
  %
  %=========
  % 英文信息
  %=========
  % etitle={An Introduction to \LaTeX{} Thesis Template of Tsinghua University v\version},
  % 这块比较复杂,需要分情况讨论:
  % 1. 学术型硕士
  %    edegree:必须为Master of Arts或Master of Science(注意大小写)
  %             “哲学、文学、历史学、法学、教育学、艺术学门类,公共管理学科
  %              填写Master of Arts,其它填写Master of Science”
  %    emajor:“获得一级学科授权的学科填写一级学科名称,其它填写二级学科名称”
  % 2. 专业型硕士
  %    edegree:“填写专业学位英文名称全称”
  %    emajor:“工程硕士填写工程领域,其它专业学位不填写此项”
  % 3. 学术型博士
  %    edegree:Doctor of Philosophy(注意大小写)
  %    emajor:“获得一级学科授权的学科填写一级学科名称,其它填写二级学科名称”
  % 4. 专业型博士
  %    edegree:“填写专业学位英文名称全称”
  %    emajor:不填写此项
  % edegree={Doctor of Engineering},
  % emajor={Computer Science and Technology},
  % eauthor={Xue Ruini},
  % esupervisor={Professor Zheng Weimin},
  % eassosupervisor={Chen Wenguang},
  % 日期自动生成,若需指定按如下方式修改:
  % edate={December, 2005}
  %
  % 关键词用“英文逗号”分割
  ckeywords={智能家居, BLE Mesh, 插件化, 开放平台},
  % ekeywords={TeX, LaTeX, CJK, template, thesis}
}
\begin{schoollogo}
  \begin{figure}[H]
    \centering
    \includegraphics{logo.png}
  \end{figure}
\end{schoollogo}
% 定义中英文摘要和关键字
\begin{cabstract}
  本项目主要实现了一个基于BLE Mesh的智能家居平台,分为BLE Mesh节点、网关端、移动端三部分,相较于现有其他智能家居其主要优势是高扩展性、高灵活性、高开放性以及低成本。对于一般用户,降低了将自己整个家庭升级为真正的智能家庭的难度;而对于具有一定技术能力的用户以及爱好者或其他智能家居厂商,则提供了一个开放的平台来更便捷地接入自己的智能设备或控制软件来实现自己的想法。

  本项目的创新点主要有:
  \begin{itemize}
    \item \textbf{节点布置灵活。}使用BLE Mesh协议作为节点间通讯的协议,使得节点布置更加方便,不必考虑传统的单对多信号覆盖问题。
    \item \textbf{低成本。}使用价格低廉的Nordic nRF52832芯片作为BLE Mesh节点,使得用户能够更低成本地使用智能家居。并具有可选的网关端,使得用户能够更低成本地构建智能家庭系统。
    \item \textbf{扩展性强。}网关使用全插件化的方案,使得具有技术能力的用户能够更好、更简单地扩展网关端功能。
    \item \textbf{原生化体验。}网关端支持Apple HomeKit协议,使得苹果设备用户能用系统原生的“家庭”应用更方便地控制整个智能家庭系统。
    \item \textbf{稳定性强。}网关端使用高性能的Go语言开发,使得网关端运行更加稳定。
    \item \textbf{跨平台统一性。}移动端使用Flutter框架开发,使得在Android/iOS平台中应用体验无太大差异。
  \end{itemize}
\end{cabstract}

% 如果习惯关键字跟在摘要文字后面,可以用直接命令来设置,如下:
% \ckeywords{\TeX, \LaTeX, CJK, 模板, 论文}

% \begin{eabstract}
%    An abstract of a dissertation is a summary and extraction of research work
%    and contributions. Included in an abstract should be description of research
%    topic and research objective, brief introduction to methodology and research
%    process, and summarization of conclusion and contributions of the
%    research. An abstract should be characterized by independence and clarity and
%    carry identical information with the dissertation. It should be such that the
%    general idea and major contributions of the dissertation are conveyed without
%    reading the dissertation.

%    An abstract should be concise and to the point. It is a misunderstanding to
%    make an abstract an outline of the dissertation and words ``the first
%    chapter'', ``the second chapter'' and the like should be avoided in the
%    abstract.

%    Key words are terms used in a dissertation for indexing, reflecting core
%    information of the dissertation. An abstract may contain a maximum of 5 key
%    words, with semi-colons used in between to separate one another.
% \end{eabstract}

% \ekeywords{\TeX, \LaTeX, CJK, template, thesis}
