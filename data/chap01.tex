\chapter{引言}

\section{研究背景}
如今,我们正处于一个万物互联的时代。小至智能手表、智能手机,大至铁路运输网、航空运输图,一切都直接或间接连接到了我们的互联网。

在我们的日常生活中,许多灯泡、开关、温湿度传感器前面也加上了“智能”。有了“智能”化的这些物件,意味着我们可以在手机上控制它们、让它们之间按照一定的规则自动调控,更加便利了我们的生活。

但实际上,我们很难能够实现真正意义上的智能:不同的智能设备多来自于不同的智能厂商,而不同厂商又拥有他们独自的控制软件,彼此不互通,导致根本不能实现自动调控;当用户要操控设备时,也要在众多控制App中选择,显得十分繁琐。

\section{研究内容}
内容

\section{研究难点}
内容
