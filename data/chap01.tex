\chapter{引言}

\section{研究背景}
如今,我们正处于一个万物互联的时代。小至智能手表、智能手机,大至铁路运输网、航空运输图,一切都直接或间接连接到了我们的互联网。

在我们的日常生活中,许多灯泡、开关、温湿度传感器前面也加上了“智能”。有了“智能”化的这些物件,意味着我们可以在手机上控制它们、让它们之间按照一定的规则自动调控,更加便利了我们的生活。

但实际上,我们很难能够实现真正意义上的智能:不同的智能设备多来自于不同的智能厂商,而不同厂商又拥有他们独自的控制软件,彼此不互通,导致根本不能实现自动调控;当用户要操控设备时,也要在众多控制App中选择,显得十分繁琐。

\section{研究内容}
我们主要通过BLE Mesh组网来实现联动。

网络只由设备组成,并不需要路由的参与。而控制端采用智能手机,给用户提供便捷的同时,也为用户节省了构建网络的成本。因为网络的延伸不需要路由的参与,所以网络也更容易进行部署。

除此之外还有一个巨大的优势,现在的智能手机、平板和电脑都配有蓝牙,用户通过蓝牙连接BLE Mesh网络,可以避免因为网络原因造成的延时和瘫痪,同时也不需要去配置复杂的网关。大大提升用户体验。

现在大部分的智能家居都需要设备连接互联网进行控制家具,而采用网状网络可以使家具的模组间互相沟通,能使无互联网的状态下也能控制家具。

\section{研究难点}
BLE Mesh的flooding协议虽然比较简单,但是它有比较明显的缺点,比如,大量的广播包转发消耗更多的网络能量,所以目前BLE Mesh的应用范围还是比较受限。

正如协议中所描述,在后续的修订版本中可能会引入基于路由协议的Mesh网络,到那时BLE Mesh的应用范围能够更广泛。
