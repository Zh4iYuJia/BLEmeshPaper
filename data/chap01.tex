\chapter{引言}

\section{研究背景}
如今,我们正处于一个万物互联的时代。小至智能手表、智能手机,大至铁路运输网、航空运输图,一切都直接或间接连接到了我们的互联网。

在我们的日常生活中,许多灯泡、开关、温湿度传感器前面也加上了“智能”。有了“智能”化的这些物件,意味着我们可以在手机上控制它们、让它们之间按照一定的规则自动调控,更加便利了我们的生活。

但实际上,我们很难能够实现真正意义上的智能:不同的智能设备多来自于不同的智能厂商,而不同厂商又拥有他们独自的控制软件,彼此不互通,导致根本不能实现自动调控;当用户要操控设备时,也要在众多控制App中选择,显得十分繁琐。

\section{研究内容}
本项目主要实现了一个通用的、高扩展性的基于BLE Mesh的智能家居平台,其分为BLE Mesh节点、网关端、移动端三部分。

本项目的BLE Mesh网络可以只由节点组成,并不强制需要网关的参与。本项目节点中采用的Nordic nRF52832芯片也具有低成本的特点,每个节点还可接入多个传感器、继电器等模块,让更多用户有可能将自己整个家庭升级为真正的智能家庭。具有一定技术能力的用户也可根据未来完全开源的PCB、代码等资源自行构建。

本项目的网关端采用了全插件化的方案,用户只需将需要的插件放至网关端插件目录中,就可以接入其他设备以及其他的控制平台,目前本项目已经开发了HomeKit插件来接入到Apple HomeKit。具有一定技术能力的用户也可根据网关端插件API自制插件,以接入自己原有的智能家居硬件或是自己的智能家居控制软件。

本项目的移动端使用Flutter框架开发,具有多平台界面逻辑的一致性,并使用了Material Design,使得界面简洁明了,软件易用性强。
\section{研究难点}
本项目涉及到的方面很多,既涉及到嵌入式设备,又涉及到网络应用开发及移动应用设计。项目的架构设计、流程设计、技术选型、代码的实现、版本管理等都具有一定挑战性。

本项目的BLE Mesh节点,需要考虑到多功能与低功耗之间的平衡,还需要对代码进行优化,以便在相对低性能的嵌入式芯片上正常地运行程序。

本项目的网关端使用了全插件化的方案,在研究过程中需要处处考虑到不同设备间的兼容问题,并且为了方便具有一定技术能力的用户自行开发新插件,还需确保插件API开放的能力足够。

本项目的移动端虽使用了Flutter框架开发,但iOS/Android平台的差异仍要单独考虑,并且调用各平台独有接口的方式变得更加复杂,有时还需要针对各平台编写对应的插件来解决调用独有接口的问题。
\section{创新点}
本项目的创新点主要有以下几点:
\begin{itemize}
    \item \textbf{节点布置灵活。}使用BLE Mesh协议作为节点间通讯的协议,使得节点布置更加方便,不必考虑传统的单对多信号覆盖问题。
    \item \textbf{低成本。}使用价格低廉的Nordic nRF52832芯片作为BLE Mesh节点,使得用户能够更低成本地使用智能家居。并具有可选的网关端,使得用户能够更低成本地构建智能家庭系统。
    \item \textbf{扩展性强。}网关使用全插件化的方案,使得具有技术能力的用户能够更好、更简单地扩展网关端功能。
    \item \textbf{原生化体验。}网关端支持Apple HomeKit协议,使得苹果设备用户能用系统原生的“家庭”应用更方便地控制整个智能家庭系统。
    \item \textbf{稳定性强。}网关端使用高性能的Go语言开发,使得网关端运行更加稳定。
    \item \textbf{跨平台统一性。}移动端使用Flutter框架开发,使得在Android/iOS平台中应用体验无太大差异。
\end{itemize}